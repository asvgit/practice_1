\documentclass[a4paper,14pt]{extarticle}
\usepackage[utf8]{inputenc}
\usepackage[english,russian]{babel}
\usepackage{indentfirst}
\usepackage{misccorr}
\usepackage{graphicx}
\usepackage{amsmath}
\usepackage{luatextra}
\setmainfont{Times New Roman}
\setmonofont{Courier New}
\usepackage[left=3cm,right=1.5cm,
    top=1.5cm,bottom=2cm,bindingoffset=0cm]{geometry}
\providecommand\No{\char"2116}%{№}%
% \usepackage{sectsty}
% \changefontsizes[21pt]{14pt}
% \sectionfont{\fontsize{14}{14}\selectfont}

\usepackage{titlesec}

\usepackage{setspace}
\newcounter{mysection}
\titleformat{\section}
  {\centering\normalfont\fontsize{16}{16}\bfseries}{}{1em}{}
\titleformat{\subsection}
  {\normalfont\fontsize{14}{14}\bfseries}{\arabic{mysection}.\arabic{subsection}}{1em}{}

\usepackage{tikz}
\usepackage{verbatim}
\usepackage{pgfplots}
\usetikzlibrary{chains, shapes.misc}
\usetikzlibrary{shapes,arrows}
\usetikzlibrary{shapes, arrows, chains}

\tikzset{
	line/.style={draw, -latex'},
	every join/.style={line},
	u/.style={anchor=south},
	r/.style={anchor=west},
	fxd/.style={text width = 6em},
	it/.style={font={\small\itshape}},
	bf/.style={font={\small\bfseries}}
}
\tikzstyle{base} =
	[
		draw,
		on chain,
		on grid,
		align=center,
		minimum height=4ex,
		minimum width = 10ex,
		node distance = 6mm and 60mm,
		text badly centered
	]
\tikzstyle{coord} =
	[
		coordinate,
		on chain,
		on grid
	]
\tikzstyle{cloud} =
	[
		base,
		ellipse,
		fill = red!5,
		node distance = 3cm,
		minimum height = 2em
	]
\tikzstyle{decision} =
	[
		base,
		diamond,
		aspect=2,
		fill = green!10,
		node distance = 2cm,
		inner sep = 0pt
	]
\tikzstyle{block} =
	[
		rectangle,
		base,
		fill = blue!3,
		rounded corners,
		minimum height = 2em
	]
\tikzstyle{print_block} =
	[
		base,
		tape,
		tape bend top=none,
		fill = yellow!10
	]
\tikzstyle{io} =
	[
		base,
		trapezium,
		trapezium left angle = 70,
		trapezium right angle = 110,
		fill = blue!5
	]
\tikzstyle{for} = 
	[
		base,
		shape=chamfered rectangle,
		chamfered rectangle xsep=2cm, 
	]
\makeatletter
\pgfkeys{/pgf/.cd,
	subrtshape w/.initial=2mm,
	cycleshape w/.initial=2mm
}
\pgfdeclareshape{subrtshape}{
	\inheritsavedanchors[from=rectangle]
	\inheritanchorborder[from=rectangle]
	\inheritanchor[from=rectangle]{north}
	\inheritanchor[from=rectangle]{center}
	\inheritanchor[from=rectangle]{west}
	\inheritanchor[from=rectangle]{east}
	\inheritanchor[from=rectangle]{mid}
	\inheritanchor[from=rectangle]{base}
	\inheritanchor[from=rectangle]{south}
	\backgroundpath{
		\southwest \pgf@xa=\pgf@x \pgf@ya=\pgf@y
		\northeast \pgf@xb=\pgf@x \pgf@yb=\pgf@y
		\pgfmathsetlength\pgfutil@tempdima{\pgfkeysvalueof{/pgf/subrtshape w}}
		\def\ppd@offset{\pgfpoint{\pgfutil@tempdima}{0ex}}
		\def\ppd@offsetm{\pgfpoint{-\pgfutil@tempdima}{0ex}}
		\pgfpathmoveto{\pgfqpoint{\pgf@xa}{\pgf@ya}}
		\pgfpathlineto{\pgfqpoint{\pgf@xb}{\pgf@ya}}
		\pgfpathlineto{\pgfqpoint{\pgf@xb}{\pgf@yb}}
		\pgfpathlineto{\pgfqpoint{\pgf@xa}{\pgf@yb}}
		\pgfpathclose
		\pgfpathmoveto{\pgfpointadd{\pgfpoint{\pgf@xa}{\pgf@yb}}{\ppd@offsetm}}
		\pgfpathlineto{\pgfpointadd{\pgfpoint{\pgf@xa}{\pgf@ya}}{\ppd@offsetm}}
		\pgfpathlineto{\pgfpointadd{\pgfpoint{\pgf@xb}{\pgf@ya}}{\ppd@offset}}
		\pgfpathlineto{\pgfpointadd{\pgfpoint{\pgf@xb}{\pgf@yb}}{\ppd@offset}}
		\pgfpathclose
	}
}
\pgfdeclareshape{cyclebegshape}{
	\inheritsavedanchors[from=rectangle]
	\inheritanchorborder[from=rectangle]
	\inheritanchor[from=rectangle]{north}
	\inheritanchor[from=rectangle]{center}
	\inheritanchor[from=rectangle]{west}
	\inheritanchor[from=rectangle]{east}
	\inheritanchor[from=rectangle]{mid}
	\inheritanchor[from=rectangle]{base}
	\inheritanchor[from=rectangle]{south}
	\backgroundpath{
		\southwest \pgf@xa=\pgf@x \pgf@ya=\pgf@y
		\northeast \pgf@xb=\pgf@x \pgf@yb=\pgf@y
		\pgfmathsetlength\pgfutil@tempdima{\pgfkeysvalueof{/pgf/cycleshape w}}
		\pgfpathmoveto{\pgfqpoint{\pgf@xa}{\pgf@ya}}
\pgfpathlineto{\pgfpointadd{\pgfpoint{\pgf@xa}{\pgf@yb}}{\pgfpoint{0ex}{-\pgfutil@tempdima}}}
\pgfpathlineto{\pgfpointadd{\pgfpoint{\pgf@xa}{\pgf@yb}}{\pgfpoint{\pgfutil@tempdima}{0ex}}}
\pgfpathlineto{\pgfpointadd{\pgfpoint{\pgf@xb}{\pgf@yb}}{\pgfpoint{-\pgfutil@tempdima}{0ex}}}
\pgfpathlineto{\pgfpointadd{\pgfpoint{\pgf@xb}{\pgf@yb}}{\pgfpoint{0ex}{-\pgfutil@tempdima}}}
\pgfpathlineto{\pgfqpoint{\pgf@xb}{\pgf@ya}}
		\pgfpathclose
	}
}
\pgfdeclareshape{cycleendshape}{
	\inheritsavedanchors[from=rectangle]
	\inheritanchorborder[from=rectangle]
	\inheritanchor[from=rectangle]{north}
	\inheritanchor[from=rectangle]{center}
	\inheritanchor[from=rectangle]{west}
	\inheritanchor[from=rectangle]{east}
	\inheritanchor[from=rectangle]{mid}
	\inheritanchor[from=rectangle]{base}
	\inheritanchor[from=rectangle]{south}
	\backgroundpath{
		\southwest \pgf@xa=\pgf@x \pgf@ya=\pgf@y
		\northeast \pgf@xb=\pgf@x \pgf@yb=\pgf@y
		\pgfmathsetlength\pgfutil@tempdima{\pgfkeysvalueof{/pgf/cycleshape w}}
		\pgfpathmoveto{\pgfqpoint{\pgf@xb}{\pgf@yb}}
\pgfpathlineto{\pgfpointadd{\pgfpoint{\pgf@xb}{\pgf@ya}}{\pgfpoint{0ex}{\pgfutil@tempdima}}}
\pgfpathlineto{\pgfpointadd{\pgfpoint{\pgf@xb}{\pgf@ya}}{\pgfpoint{-\pgfutil@tempdima}{0ex}}}
\pgfpathlineto{\pgfpointadd{\pgfpoint{\pgf@xa}{\pgf@ya}}{\pgfpoint{\pgfutil@tempdima}{0ex}}}
\pgfpathlineto{\pgfpointadd{\pgfpoint{\pgf@xa}{\pgf@ya}}{\pgfpoint{0ex}{\pgfutil@tempdima}}}
\pgfpathlineto{\pgfqpoint{\pgf@xa}{\pgf@yb}}
		\pgfpathclose
	}
}
\makeatother
\tikzstyle{subroutine} =
	[
		base,
		subrtshape,
		fill = green!25
	]
\tikzstyle{cyclebegin} =
	[
		base,
		cyclebegshape,
		fill = blue!25
	]
\tikzstyle{cycleend} =
	[
		base,
		cycleendshape,
		fill = blue!25
	]
\tikzstyle{connector} =
	[
		base,
		circle,
		fill = red!25
	]

\usepackage{listings}
\usepackage{xcolor}
\lstset { %
    language=C++,
    backgroundcolor=\color{black!5}, % set backgroundcolor
    basicstyle=\footnotesize,% basic font setting
}
\lstdefinestyle{customc}{
  belowcaptionskip=1\baselineskip,
  breaklines=true,
  frame=L,
  xleftmargin=\parindent,
  language=C,
  showstringspaces=false,
  basicstyle=\footnotesize\ttfamily,
  keywordstyle=\bfseries\color{green!40!black},
  commentstyle=\itshape\color{purple!40!black},
  identifierstyle=\color{blue},
  stringstyle=\color{orange},
}

\lstdefinestyle{customasm}{
  belowcaptionskip=1\baselineskip,
  frame=L,
  xleftmargin=\parindent,
  language=[x86masm]Assembler,
  basicstyle=\footnotesize\ttfamily,
  commentstyle=\itshape\color{purple!40!black},
}

\lstset{escapechar=@,style=customc}

\usepackage{hyperref,xcolor}

\begin{document}
\begin{titlepage}
	\newpage
	\begin{center}
		ФЕДЕРАЛЬНОЕ АГЕНСТВО ЖЕЛЕЗНОДОРОЖНОГО ТАРНСПОРТА \\
		\vspace{14pt}
		Федерально государственное бюджетное образовательное учреждение \\\vspace{7pt} высшего образования \\\vspace{7pt}
		<<Иркутский государственный университет путей и сообщения>> \\\vspace{7pt}
		(ФГБОУ ВО ИрГУПС) \\\vspace{7pt}
		Факультет <<Управление на транспорте и информационнные технологии>> \\\vspace{7pt}
		Кафедра <<Информационные системы и защита информации>>
	\end{center}
	\vspace{42pt}
	\begin{center}
		ОТЧЕТ ПО ПРАКТИКЕ
	\end{center}
	\vspace{-14pt}
	\begin{center}
		Учебная - по получению первичных профессиональных умений и навыков (предметно - исследовательская)\\
	\vspace{14pt}
		УП.430200.090404.000.ПЗ
	\end{center}
	\vspace{56pt}
	\begin{flushleft}
		\begin{tabular}{p{0.57\textwidth}l}
			Выполнил:
				&	Проверил: \\
			студент группы ПИм.1-16-1, Арляпов С.В.
				&	доцент Звонков И.В.\\
			Шифр: 1621345
				&	\_\_\_\_\_\_\_\_\_\_\_\_\_\_\_  \\
			&\\
			&\\
			<<\_\_\_>>\_\_\_\_\_\_\_\_\_\_\_\_\_\_\_20\_\_г.
				&	<<\_\_\_>>\_\_\_\_\_\_\_\_\_\_\_\_\_\_\_20\_\_г.
		\end{tabular}
	\end{flushleft}
	\vspace{\fill}
	\begin{center}
		Иркутск 2017
	\end{center}
\end{titlepage}


\setcounter{page}{2}
\tableofcontents⊧

\newpage
\section{Задание на практику }
В результате прохождения практики необходимо изучить предметную область для дальнейших исследований.

В ходе практики должны быть освоены компетенции способность совершенствовать и развивать свой интеллектуальный и общекультурный уровень, а также способность проектировать распределенные информационные системы, их компоненты и протоколы их взаимодействия.

\newpage
\section{Введение}

Согласно  поределению, данному выцдающемся учёным А.М. Летовым, стоящим у истоков современной теории управления и сделавшим многое для её развития, теория управления <<есть совокупность методов позволяющих выработать и обосноваь решение, которое применяется для достижения заранее поставленной целию в условиях каких-либо определённой ситуаци>>. В частности, теория автоматического управления -- <<наука ометодах ометодах определения законов управления какими-либо объектами, допускающих реализацию с поощью технических средств автоматики>>[1].

    Книга <<Интелектное управление динамическими системами>> посвящена методам интелектного управления и интелектного анализа и проектирования динамических упарвляемых систем, объектами которых является конкретный класс систем -- движущиеся объекты технической природы. 3 глава этой книги посвящена применению процедур автоматизации вывода как альтернативе освоеннм в инженерной практике алгоритмическим процедурам. Демонстрация применения логических языков для представления знаний и интелектного упраления приведена в главе 3 на примере упраления группой пассажирских лифтов. 

    Извыше описанной книги берётся задача реализовать интелектное управление группой пассажирских ливтов на основе системы автоматического доказательства торем (АДТ).

\subsection{Основная часть}

Одним из подходов к интеллектуализации программных систем является разработка алгоритмов обработки информации, основанных на моделировании процесса рассуждений. Наиболее формализованные подходы базируются на автоматическом построении логического вывода (ЛВ) в некоторой системе формализованных знаний (логического описания предметной области). Программные системы для поиска ЛВ называют системами автоматического доказательства теорем (АДТ), поскольку утверждения, для которых существует ЛВ, являются теоремами (в заданном исчислении)[2].

Формализации некоторых предметных областей, например, аксиоматизации математических теорий, верификация программного и аппаратного обеспечения, являются весьма громоздкими, и использование систем АДТ для построения ЛВ становится необходимым инструментом исследования. Кроме того, системы АДТ используются в качестве механизма ЛВ, например, в системах планирования действий, системах поддержки принятия решений, экспертных системах, где важным фактором является время решения задачи и другие критерии. Далее, в работе, под «поиском ЛВ», как правило, будем иметь ввиду автоматизированный поиск ЛВ.

Автоматическое доказательство теорем с обоснованием его эффективности широко используется в таких областях как верификация программных и аппаратных систем, синтез программного обеспечения, решение проблем математики, логическое программирование, дедуктивные базы данных и др[2].

\subsection{Позитивно-образованные формулы}

Позитивно-образованными формулами (ПО-формылами, ПОФ) называется вид формул, для записи которых используются только позитивные типовые кванторы \forall{}  и \exists:

Пусть X — множество переменных, и A — конъюнкт.

1. $\exists_x A$ и $\forall_x A$ есть \exists-ПОФ и \forall-ПОФ соответственно.

2. Если F = \{$F_1$,…,$F_n$\} есть \forall-ПОФы, тогда  ∶ F есть \exists-ПОФ.

3. Если F = \{$F_1$,…,$F_n$\} есть \exists-ПОФы, тогда  ∶ F есть \forall-ПОФ.

4. Любая \exists-ПОФ или \forall-ПОФ есть ПОФ.

Данные формулы не содержат операторов отрицания. Также ПО-формула является особым видом записи классических формул языка предикатов, подобно КНФ, ДНФ и др., поскольку любая формула языка предикатов первого порядка представима как позитивно-образованная формула.

Канонический вид ПО-формулы начинается с \forall\emptyset. Очевидно, что любая ПОФ приводима к каноническому виду. 

Типовые кванторы \forall\emptyset и \exists\emptyset называются фиктивными, поскольку не влияют на истинность формулы и не связывают никаких переменных, а только лишь служат конструкциями сохраняющими корректную запись ПО-формулы.

Для удобства ПО-формулы представляются в древовидной форме:

$Q_x A: $\{$F_1,...,F_n$\}$ \equiv Q_x A:  $
\begin{math}
    \begin{cases}
        F_1\\
        ...\\
        F_n
    \end{cases},
\end{math}

 где Fi – ПО-формула, А – набор атомарных формул, Q некоторый квантор, который отличен от кванторов в начале формул F.

 Некоторые части канонической ПО-формулы имеют специальные названия:

 1. Корневой узел \forall\emptyset называется корнем ПО-формулы;

 2. Дочерние узлы корня ПО-формулы имеют вид $\exists_x A$ и называются базами ПО-формулы, конъюнкт А называется базой фактов, а вся подформула начинающаяся с базового узла называется базовой подформулой;

 3. Дочерние узлы баз имею вид $\forall_x B$ и называются вопросами к родительской базе. Если вопрос является листовым узлом $\forall_x B\equiv\forall_x B:false$, то он называется целевым вопросом.

 4. Поддеревья вопросов называются консеквентами или следствиями. Следствием целевого вопроса является false.

\subsection{Система АДТ PRISNIF}

В ИДСТУ СО РАН А.А. Ларионов и Е.А.Черкашин занимаются исследованием поиска  ЛВ на базе ПО-формул. Основным результатом исследований является программная система АДТ PRISNIF. Применение системы для решения ряда задач из библиотеки TPTP показывает, что система соответствует мировому уровню в данной области.

Сама система реализована на языке программирования D, но новая версия системы АДТ разрабатывается на языке программирования Rust.

Стоит подчеркнуть, что система АДТ PRISNIF реализует поиск ЛВ на базе ПО-формул. Этот факт, задаёт к реализации программных модулей некоторый условия, которые учитывались в разработке программного продукта.

Также стоит отметить, что в ИДСТУ СО РАН лаборатория Информационно-управляющих систем нуждается в подобной системе для использования её в качестве автоматической системы управления.

\subsection{SymPy}
SymPy представляет собой открытую библиотеку символьных вычислений на языке Python. Цель SymPy - стать полнофункциональной системой компьютерной алгебры (CAS), при этом сохраняя код максимально понятным и легко расширяемым. SymPy полностью написан на Python и не требует сторонних библиотек.

SymPy можно использовать не только как модуль, но и как отдельную программу. Программа удобна для экспериментов или для обучения. Она использует стандартный терминал IPython, но с уже включенными в нее важными модулями SymPy и определенными переменными x, y, z.

\newpage
\section{Заключение}

Используя выше представленные материалы, можно попытаться реализовать интелектно управляемую динамическую систему на основе системы автоматического доказательства теорем на базе поитивно образованных формул.

\newpage

\section{Источники}

1. С. Н. Васильев. Интеллектуальное управление динамическими системами / С. Н. Васильев, А. К. Жерлов, Е. А. Федосов, Б. Е. Федунов - М.. Физико-математическая литература, 200. - 352 с.

2. А. А. Ларионов. Программные технологии для эффективного поиска логического вывода в исчислении позитивно-образованных формул / А. А. Ларионов, Е. А. Черкашин – Иркутск : Изд-во ИГУ, 2013. – 104 c.

3. \url{http://www.asmeurer.com}
\end{document}
